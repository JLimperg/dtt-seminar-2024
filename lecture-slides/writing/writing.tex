\documentclass[xetex]{beamer}
\usepackage{fontspec}
\usepackage{unicode-math}
\usepackage{csquotes}

\setmainfont[Scale=MatchLowercase]{XITS}
\setmathfont[Scale=MatchLowercase]{XITS Math}
\setsansfont{DejaVu Sans}
\setmonofont{DejaVu Sans Mono}

\setbeamertemplate{navigation symbols}{}
\setbeamertemplate{itemize item}{\Large\textbullet}
\setbeamertemplate{itemize subitem}{\Large\textbullet}
\setbeamertemplate{itemize subsubitem}{\Large\textbullet}

\setlength{\tabcolsep}{0.5em}

\AtBeginSection[]
{
  \begin{frame}
    \tableofcontents[currentsection, hideallsubsections]
  \end{frame}
}

\AtBeginSubsection[]
{
  \begin{frame}
    \tableofcontents[sectionstyle=show/hide, subsectionstyle=show/shaded/hide, subsubsectionstyle=hide/hide/hide]
  \end{frame}
}

\newcommand{\bad}[1]{\textcolor{orange}{#1}}
\newcommand{\good}[1]{\textcolor{blue}{#1}}

\hypersetup{
  colorlinks=true,
  allcolors=.,
  urlcolor=cyan
}

\begin{document}

\title{Writing a Seminar Paper}
\author{Jannis Limperg\inst{1}}
\institute{%
  \inst{1} University of Munich (LMU), \href{mailto:jannis@limperg.de}{jannis@limperg.de}}
\date{22nd May 2024}

\begin{frame}
  \maketitle
\end{frame}

\begin{frame}
  \tableofcontents[hideallsubsections]
\end{frame}

\section{Formal Requirements}

\begin{frame}
  \frametitle{Formal Requirements}

  20000--30000 characters (ca.\ 5--8 pages), excluding bibliography and appendices

  \medskip
  \pause

  German or English
\end{frame}

\section{Content}

\begin{frame}
  \frametitle{Goal}

  Officially: summarise one or more papers about a given topic for your fellow students.

  \medskip
  \pause

  Inofficially: convince me that you have thoroughly understood the topic.
\end{frame}

\begin{frame}
  \frametitle{Content}

  Main ideas from the papers.

  \medskip
  \pause

  Examples, ideally novel.
\end{frame}

\begin{frame}
  \frametitle{Target Audience}

  The target audience is your fellow students.

  \medskip
  \pause

  They are familiar with
  \begin{itemize}
    \item things usually taught in a computer science BSc
    \item topics discussed in previous presentations (see Moodle)
  \end{itemize}
\end{frame}

\section{Process}

\begin{frame}
  \frametitle{Process}

  \begin{enumerate}[<+->]
    \item fix a title and table of contents
    \item first draft
          \begin{itemize}
            \item can be very rough
            \item start from the top \emph{or} start with the main content
          \end{itemize}
    \item revise/refactor
          \begin{itemize}
            \item big changes: add or remove content, split a section, move a section, move content between sections
            \item small changes: rewrite a sentence or paragraph, split/join sentences or paragraphs, change wording
            \item most time spent here!
          \end{itemize}
    \item get feedback, revise again
    \item treat this as an exercise for your MSc.\ thesis (and all your future emails, chats, technical docs, \dots)
  \end{enumerate}
\end{frame}

\begin{frame}
  \frametitle{Peer Feedback}

  Pair up and give feedback on each other's drafts.

  \medskip
  \pause

  Receiving and giving(!) careful feedback helps a lot.

  \medskip
  \pause

  After the lecture, I'll make a thread on Zulip where you can pair up.
\end{frame}

\begin{frame}
  \frametitle{Curse of Knowledge}

  \begin{quote}
    The main cause of incomprehensible prose is the difficulty of imagining what it's like for someone else not to know something that you know.

    --- Steven Pinker, The Sense of Style
  \end{quote}

  \pause
  \medskip

  Symptoms:
  \begin{itemize}[<+->]
    \item omitting important background information
    \item not defining unfamiliar terms
    \item jumping to conclusions
    \item missing logical links
    \item lack of examples, overly abstract prose
    \item \dots
  \end{itemize}
\end{frame}

\begin{frame}
  \frametitle{Curse of Knowledge}

  Partial solutions:
  \begin{itemize}[<+->]
    \item get feedback from your target audience
    \item systematically check for the mentioned symptoms
  \end{itemize}
\end{frame}

\section{Structure}

\subsection{Document Level}

\begin{frame}
  \frametitle{Document Level}

  \begin{enumerate}
    \item Titlepage
    \item Table of Contents
    \item Introduction (0.5--1 pages)
    \item Background (0.5--1 pages)
    \item Main Idea 1
    \item \dots
    \item Main Idea $n$
    \item Conclusion (0.5 pages)
    \item Acknowledgements (1 paragraph, optional)
    \item Bibliography (does not count towards character limit)
  \end{enumerate}
\end{frame}

\begin{frame}
  \frametitle{Titlepage}

  \begin{itemize}
    \item title of paper
    \item author
    \item title of seminar
    \item date
  \end{itemize}
\end{frame}

\begin{frame}
  \frametitle{Introduction}

  Which problems does the paper address? Short examples help immensely.

  \medskip
  \pause

  Why are the problems important?

  \medskip
  \pause

  Briefly: what are the presented solutions to these problems?
\end{frame}

\begin{frame}
  \frametitle{Background}

  What background information is necessary for your audience to understand the main ideas?

  \medskip
  \pause

  Do not include material that
  \begin{itemize}
    \item is not necessary to understand the main ideas or
    \item is better explained at the point where it is used.
  \end{itemize}
\end{frame}

\begin{frame}
  \frametitle{Main Ideas}

  Select the most important ideas from the papers.

  \medskip
  \pause

  For each idea, describe the problem and the solution in as much detail as possible.

  \medskip
  \pause

  Give examples. Your own examples are worth more than examples taken from the papers.
\end{frame}

\begin{frame}
  \frametitle{Conclusion}

  Very briefly summarise the main ideas again.
\end{frame}

\begin{frame}
  \frametitle{Acknowledgements}

  Briefly acknowledge everyone who substantially helped you.

  \medskip

  If you used AI tools, briefly describe how and for what.
\end{frame}

\subsection{Paragraph Level}

\begin{frame}
  \frametitle{Topic Sentences}

  Each paragraph should have one topic.

  \medskip

  The first few sentences of a paragraph announce the topic.
  They are called \emph{topic sentences}.
  The remainder of the paragraph expands on the topic.

  \medskip

  The topic sentences alone should provide a decent outline of the section.
\end{frame}

\begin{frame}
  \frametitle{Topic Sentences: Example}

  \small
  Austria-Hungary [\dots] had by contrast been self-sufficient in the major foodstuffs before 1914.
  The fact that wartime deprivation was greater than in Germany [\dots] thus requires some explanation.
  Three factors were responsible for negating the initial Habsburg advantage.
  First, the foundations of the disaster were laid already in 1914, with the Russian invasion of Galicia and Bukovina.

  [\dots]

  Second, agriculture in the rest of the Empire suffered similar problems to those experienced by farmers in Germany [\dots].

  [\dots]

  Hungary's agriculture was less badly damaged [\dots].

  [\dots]

  The third factor [\dots] was the lack of solidarity between the two halves of the Empire.\footnote{Alexander Watson, Ring of Steel}
\end{frame}

\subsection{Sentence Level}

\begin{frame}
  \frametitle{Flow}

  Each sentence in a paragraph should have a logical link to the previous sentence.

  \medskip
  \pause

  \bad{Bad:}

  {
    \small
    There are a number of efficient sorting algorithms.
    Recursion is easy to reason about, so merge sort is particularly straightforward to get right.\footnote{Example from \href{https://docs.google.com/document/d/1_vBXbugoLjO171w3wovs3ugmRQI-O6EcSVFDBF7eUzE/edit}{Benjamin C. Pierce and Rajeev Alur, Writing and Speaking with Style}}
  }

  \medskip
  \pause

  \good{Better:}

  {
    \small
    There are a number of efficient sorting algorithms.
    Of these, merge sort is particularly straightforward to get right, since recursion is easy to reason about.
  }
\end{frame}

\begin{frame}
  \frametitle{Logical Linkage: Example}

  {
  \small
  Second, agriculture in the rest of the Empire suffered \textbf{similar} problems to those experienced by farmers in Germany, not only making it impossible to replace Galician production, but actually resulting in an \textbf{even larger} food deficit.
  There was \textbf{the same} shortage of animal and human labour: the Habsburg army took 814,000 horses, about a fifth of all those in the country, on mobilization.
  Millions of men were conscripted.
  The dung and fertilizer needed to regenerate the soil were \textbf{also} in short supply.
  Statistics for food production in the region that at the war's end became the Austrian Republic illustrate how severely war affected even land untouched by military action [\dots].
  }

  \medskip

  Other linking words: however, nonetheless, further, in contrast, \dots
\end{frame}

\begin{frame}
  \frametitle{Topic and Stress}

  Start of a sentence: \emph{topic position}.
  Announces the topic of the sentence; often connects to previous material.

  \medskip

  End of a sentence: \emph{stress position}.
  Contains the most important new information.
\end{frame}

\begin{frame}
  \frametitle{Topic and Stress: Example}

  \small
  Second, \underline{agriculture in the rest of the Empire} suffered similar problems to those experienced by farmers in Germany, not only making it impossible to replace Galician production, but actually resulting in an \textbf{even larger food deficit}.
  There was \underline{the same shortage of animal and human labour}: the Habsburg army took 814,000 horses, about a fifth of all those in the country, \textbf{on mobilization}.
  \textbf{Millions of men were conscripted}.
  The \underline{dung and fertilizer} needed to regenerate the soil were also in \textbf{short supply}.
  \underline{Statistics for food production} in the region that at the war's end became the Austrian Republic illustrate how severely war affected \textbf{even land untouched by military action} [\dots].\footnote{Alexander Watson, Ring of Steel}
\end{frame}


\subsubsection{Sentence Structure}

\begin{frame}
  \frametitle{Sentence Length}

  Prefer short sentences.

  \medskip

  Use some longer sentences for variety.
\end{frame}

\begin{frame}
  \frametitle{Subject-Verb Separation}

  The subject should be followed as soon as possible by its verb.

  \medskip
  \pause

  \bad{Bad:}

  {
    \small
    The smallest of the URF's (URFA6L), a 207-nucleotide (nt) reading frame overlapping out of phase the NH2-terminal portion of the adenosinetriphosphatase (ATPase) subunit 6 gene has been identified as the animal equivalent of the recently discovered yeast H+-ATPase subunit 8 gene.\footnote{Example from Gopen and Swan, The Science of Scientific Writing.}
  }

  \medskip
  \pause

  \good{Better:}

  {
    \small
    The smallest of the URF's \good{is} URFA6L, a 207-nucleotide (nt) reading frame overlapping out of phase the NH2-terminal portion of the adenosinetriphosphatase (ATPase) subunit 6 gene.
    This URF \good{has been identified} as the animal equivalent of the recently discovered yeast H+-ATPase subunit 8 gene.
  }
\end{frame}

\begin{frame}
  \frametitle{Dependent Clauses}

  Avoid long or nested dependent clauses (\emph{Nebensätze}).

  \medskip
  \pause

  \bad{Bad:}

  {
    \small
    The observation that Dijkstra’s algorithm might be implemented using a priority queue is of note to computer scientists because it represents a significant opportunity to improve performance.\footnote{Example from Pierce and Alur.}
  }

  \medskip
  \pause

  \only<3>{%
  \good{Better:}

  {
    \small
    Dijkstra's algorithm might be implemented using a priority queue.
    This is noteworthy to computer scientists because it represents a significant opportunity to improve performance.
  }
  }
  \only<4>{%
  \good{Even better:}

  {
    \small
    Dijkstra's algorithm can be implemented with a priority queue.
    This greatly improves its performance.
  }
  }
\end{frame}


\section{Style}

\begin{frame}
  \frametitle{Concision}

  \begin{quote}
    Omit needless words.

    --- William Strunck Jr.
  \end{quote}

  \medskip
  \pause

  \bad{Bad:}

  A moderate amount of repetition is fine.

  \good{Better:}

  Some repetition is fine.

  \medskip
  \pause

  \bad{Bad:}

  For a summary, it's fine if the reference list contains exactly one work.

  \good{Better:}

  The bibliography of a summary may well contain only one entry.
\end{frame}

\begin{frame}
  \frametitle{Active and Passive}

  Prefer the active voice over the passive voice.

  \medskip
  \pause

  \bad{Bad:}

  There is opposition among many voters to nuclear power plants based on X.\footnote{Example from Pierce and Alur.}

  \medskip
  \pause

  \good{Better:}

  Many voters oppose nuclear power plants based on X.
\end{frame}

\begin{frame}
  \frametitle{Nominalisation}

  Prefer verbs over nouns.
  Humans like to read about people doing things.

  \pause
  \medskip

  \bad{Bad:}

  Premature optimisation diagnoses are difficult to get right.\footnote{Examples from Pierce and Alur.}

  \medskip
  \pause

  \good{Better:}

  It is difficult to diagnose premature optimisation.

  \pause
  \medskip

  \bad{Bad:}

  The design of the new roller coaster was more of a struggle for the engineers than had been their expectation.

  \medskip
  \pause

  \good{Better:}

  The engineers struggled more than expected with the design of the new roller coaster.
\end{frame}

\begin{frame}
  \frametitle{First Person}

  First person is fine.
  \begin{itemize}
    \item \emph{I} summarise
    \item to \emph{my} knowledge
  \end{itemize}

  \medskip
  \pause

  If you use first person a lot, stop describing your journey and describe the chapter instead.
\end{frame}

\begin{frame}
  \frametitle{First Person Plural}

  First person plural is often used to refer collectively to the author and the reader.
  \begin{itemize}
    \item \emph{we} see
    \item \emph{we} can conclude
  \end{itemize}

  \medskip
  \pause

  But only use this when it makes sense. Not:
  \begin{itemize}
    \item in this paper, \emph{we} summarise
  \end{itemize}
\end{frame}

\begin{frame}
  \frametitle{Second Person}

  Do not use second person, referring to the reader as \emph{you}.
\end{frame}

\begin{frame}
  \frametitle{Repetition}

  Some repetition is fine.
  If you use different terms for the same thing, you may create more confusion than variety.

  \medskip
  \pause

  \bad{Bad:}

  {
    \small
    Merge sort is a sorting \bad{algorithm}.
    The \bad{procedure} is implemented by recursion.
    This means the \bad{technique} requires certain compiler optimisations to become efficient.
  }

  \medskip
  \pause

  \good{Still bad but less confusing:}

  {
    \small
    Merge sort is a sorting algorithm.
    The algorithm is implemented by recursion.
    This means the algorithm requires certain compiler optimisations to become efficient.
  }
\end{frame}

\begin{frame}
  \frametitle{These Are All Just Rules of Thumb}

  Omitting too many words hurts clarity

  \medskip
  \pause

  Passive constructions can direct the reader's focus

  \medskip
  \pause

  Excessive repetition is inelegant
\end{frame}

\section{Formatting}

\begin{frame}
  \frametitle{Formatting}

  See slides by David Sabel (translated by Luca Maio) \href{https://github.com/JLimperg/dtt-seminar-2024/}{in the course repository}.
\end{frame}

\begin{frame}
  \frametitle{Use \LaTeX}

  \begin{itemize}[<+->]
    \item de facto standard in all mathematical sciences
    \item takes care of mundane formatting (references, figures, code highlighting, \dots)
    \item looks good out of the box
    \item best-in-class mathematical typesetting
    \item generates table of contents and bibliography
    \item built-in citation management
    \item easily versioned with Git
    \item infinitely extensible (if you can stomach the programming language)
  \end{itemize}

  \medskip
  \pause

  TCS provides a \href{https://www.tcs.ifi.lmu.de/lehre/bsc-master-arbeiten_de.html}{template for BSc and MSc theses} which you can adapt.
\end{frame}

\begin{frame}
  \frametitle{Title Case}

  Headings Should Use the So-Called Title Case.

  \medskip

  Very roughly: capitalise every word except \enquote*{minor words}: of, and, the, etc.
  Different style guides disagree on what exactly is a minor word.

  \medskip
  \pause

  Use \url{titlecaseconverter.com}.
\end{frame}

\begin{frame}
  \frametitle{Accessibility}

  Do not convey information exclusively through colour.

  \medskip
  \pause

  Write useful captions for graphical elements.
\end{frame}

\section{Academic Integrity}

\begin{frame}
  \frametitle{Academic Integrity}

  IFI guidelines (only in German for some reason):

  \medskip

  \url{www.medien.ifi.lmu.de/lehre/Plagiate-IfI.pdf}
\end{frame}

\begin{frame}
  \frametitle{Citations}

  Cite every paper/document that introduces an idea which is not common knowledge and which is used in your work.

  \medskip
  \pause

  Do not cite papers just to pad your bibliography.
  The bibliography of a summary may well contain only one entry.
\end{frame}

\begin{frame}
  \frametitle{Quotes}

  Always cite the source.

  \medskip
  \pause

  For direct (word-for-word) quotes: use \enquote{quotation marks} or typeset them as block quotes.

  \medskip
  \pause

  For indirect (paraphrased) quotes:
  \begin{itemize}
    \item Make it clear what part of the text is an indirect quote.
    \item Cite the source once per indirect quote.
    \item In a summary: okay to cite the summarised work only once.
  \end{itemize}
\end{frame}

\begin{frame}
  \frametitle{Editing}

  You may let other people edit your drafts.

  \medskip
  \pause

  You may not let other people write parts of your text.

  \medskip
  \pause

  Editors should be acknowledged.
\end{frame}

\begin{frame}
  \frametitle{AI Tools (ChatGPT)}

  Generative AI tools are allowed, but the text must remain substantially yours.

  \medskip
  \pause

  You must describe how you used AI tools in an acknowledgement.

  \medskip
  \pause

  AI tools may be useful for drafting and to improve your English.
  But beware of factual errors, imprecision and waffling.
\end{frame}

\section{Additional Resources}

\begin{frame}
  \frametitle{Additional Resources}

  \small

  \good{Benjamin Pierce and Rajeev Alur, \href{https://docs.google.com/document/d/1_vBXbugoLjO171w3wovs3ugmRQI-O6EcSVFDBF7eUzE/edit}{Writing and Speaking with Style}}

  Lecture notes for a writing course for computer scientists.
  Covers many of the themes I've discussed in more detail.
  Nice exercises.

  \medskip

  \good{Steven Pinker, The Sense of Style}

  Book about nonfiction writing.
  Pierce and Alur's slides are partly based on this book, as are mine.

  \medskip

  \good{Joseph M.\ Williams and Joseph Bizup, Style: Lessons in Clarity and Grace}

  Another book about nonfiction writing.
  Recommended by Pierce and Alur for its exercises.

  \medskip

  \good{George D.\ Gopen and Judith A.\ Swan, \href{https://pages.ucsd.edu/~scoulson/101b/Science-of-Writing.pdf}{The Science of Scientific Writing}}

  Paper about some typical sentence- and paragraph-level issues.
  Excellent examples.
\end{frame}

\begin{frame}
  \frametitle{Conclusion}

  \good{Draft, then revise.} Treat writing as an optimisation problem.

  \medskip
  \pause

  Look out for \good{common issues}: subject-verb separation, logical linkage, topic and stress, excessive passive or nominalisation, etc.

  \medskip
  \pause

  Always think about the \good{target audience}.

  \medskip
  \pause

  Adhere to the \good{formal requirements}. Do not accidentally \bad{plagiarise}.
\end{frame}


\section*{Bonus: Giving a Seminar Talk}

\begin{frame}
  \frametitle{Bonus: Giving a Seminar Talk}

  Focus on the \good{main ideas}

  \medskip
  \pause

  \good{Declutter} your slides

  \medskip
  \pause

  \good{Images} are better than text

  \medskip
  \pause

  Direct the \good{listeners' attention}

  \medskip
  \pause

  \good{Rehearsing} is a superpower

  \bigskip
  \pause

  See \href{https://docs.google.com/presentation/d/1T-w7SEXwFf3HxRMYF_EnRnAMT1_sfq6TrOwV7R6amHk/edit?usp=sharing}{these slides} for more.
\end{frame}

\end{document}
